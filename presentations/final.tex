\documentclass{beamer}
\usepackage[letterspace=110]{microtype}
\usepackage[bitstream-charter]{mathdesign}
\usepackage[light]{sourcecodepro}
\usetheme{metropolis}
\setbeamercolor{normal text}{fg=black,bg=white}
\setbeamercolor{title separator}{fg=red,bg=white}
\setbeamerfont{frametitle}{family=\ttfamily\lsstyle,series=\mdseries,size=\Large}
\setbeamerfont{title}{family=\ttfamily\lsstyle,series=\mdseries,size=\huge}

\let\theframetitle\frametitle
\renewcommand\frametitle[1]{\theframetitle{\MakeUppercase{#1}}}

\usepackage[utf8]{inputenc}
\title{\texttt{tell me to survive}}
\author{David Li, Michael Mauer, and Andy Jiang}
\institute{Cornell University}
\date{May 17, 2016}

\begin{document}

\frame{\titlepage}

\begin{frame}
\frametitle{Motivation}

% I'm adding some comments to elucidate what I'm thinking about saying
% for each of these slides

\begin{itemize}[<+->]
% So we have to teach people how do OOP ("summative"...?)
% paradigm shift is hard
\item Object-oriented programming?
% The issue is in teaching to think in classes and objects ("formative"...?)
\item Object-oriented thinking!
% Comparing to the real world is a good way to explain. logical.
\item Concepts from real world examples
% The trick is to go from concepts to to code, which is concreteness fading
\item Concepts to code?
\end{itemize}
\end{frame}

% Listing things looked bad for the slides. So I wrote out concepts.
% basic idea is, there's stuff with components of our game in isolation
% but not together.
\begin{frame}
\frametitle{Related Work}
\begin{itemize}
% in fact, this is a pretty standard appraoch!
\item<1-> Visual Programming
  \begin{itemize}
  \item Scratch, Looking Glass, CodeSpells, Blockly Games, LightBot,
    Human Resource Machine, …
  \end{itemize}
% Blockly Games (but not one coherent game)
\item<2-> Concreteness Fading
  \begin{itemize}
  \item Blockly Games
  \end{itemize}
\item<3-> Object-Oriented Programming
  \begin{itemize}
  \item Looking Glass, CodeSpells
  \end{itemize}
% LightBot is hardly relevant separately, "robots" is just a theme.
\end{itemize}
\end{frame}

\begin{frame}
\frametitle{Related Work}
\begin{itemize}[<+->]
\item Combination is key
\end{itemize}
\end{frame}

\begin{frame}
\frametitle{Final Project}
\begin{itemize}[<+->]
% Maybe the concept progression could go here
\item Idea: solve puzzles with object-oriented code
\item Scope: OOP fundamentals
% TODO: ease transition from what to what?
\item Goal: ease transition
\end{itemize}
\end{frame}

\begin{frame}
\frametitle{Final Project}
Force use of OOP via:
\begin{itemize}
% Verbally cover the examples from the write-up
\item Limited block complexity
\item Block limit
\end{itemize}
\end{frame}

% Okay so it's kinda weird to give this it's own slide title
% but I thought putting on to the slide didn't look very good
% besides, despite more bullet points the previous section might
% take more time
\begin{frame}
\frametitle{Concreteness Fading}
Three stages
\begin{enumerate}
% Probably want pictures for this part...?
\item Textual description
\item Python syntax on blocks
\item Code editor
\end{enumerate}
\end{frame}

\begin{frame}
\frametitle{Concreteness Fading}
Considerations
\begin{itemize}
\item Fading + new concept = confusion
\item Scaffolding
\end{itemize}
\end{frame}

\begin{frame}
\frametitle{User Study}
\begin{itemize}
\item ``Robot Commander Aptitude Test''
\item 5 questions
\item Mix of syntax and concepts
% Once we have a definite number of results, this might
% be the place to say how many we tested on
% or that can go in results.
\end{itemize}
\end{frame}

\begin{frame}
\frametitle{Results}
\end{frame}

\begin{frame}
\frametitle{Conclusions}
\end{frame}
\end{document}
